%#################################
%#####      LSTLISTINGS      #####
%#################################


\lstset{ 
	backgroundcolor=\color{white},   % choose the background color; you must add \usepackage{color} or \usepackage{xcolor}; should come as last argument
	basicstyle=\footnotesize,        % the size of the fonts that are used for the code
	breakatwhitespace=false,         % sets if automatic breaks should only happen at whitespace
	breaklines=true,                 % sets automatic line breaking
	captionpos=b,                    % sets the caption-position to bottom
	commentstyle=\color{green!60!black},      % comment style
	deletekeywords={...},            % if you want to delete keywords from the given language
	escapeinside={\%*}{*)},          % if you want to add LaTeX within your code
	extendedchars=true,              % lets you use non-ASCII characters; for 8-bits encodings only, does not work with UTF-8
	frame=all,	                 % adds a frame around the code
	keepspaces=true,                 % keeps spaces in text, useful for keeping indentation of code (possibly needs columns=flexible)
	keywordstyle=\color{violet},     % keyword style
	language=C,                 	 % the language of the code
	morekeywords={*,..., fi, do, od, return, then},            % if you want to add more keywords to the set
	numbers=left,                    % where to put the line-numbers; possible values are (none, left, right)
	numbersep=5pt,                   % how far the line-numbers are from the code
	numberstyle=\tiny\color{gray},   % the style that is used for the line-numbers
	rulecolor=\color{black!50},         % if not set, the frame-color may be changed on line-breaks within not-black text (e.g. comments (green here))
	showspaces=false,                % show spaces everywhere adding particular underscores; it overrides 'showstringspaces'
	showstringspaces=false,          % underline spaces within strings only
	showtabs=false,                  % show tabs within strings adding particular underscores
	stepnumber=1,                    % the step between two line-numbers. If it's 1, each line will be numbered
	stringstyle=\color{blue},        % string literal style
	tabsize=2,	                     % sets default tabsize to 2 spaces
	title=\lstname                   % show the filename of files included with \lstinputlisting; also try caption instead of title
}


\lstdefinelanguage{ROSBASH}{
	keywords={catkin_make, rospack, roscd, rosls, catkin_create_pkg, roscore, rosnode, rosrun, rqt_graph, rostopic, rosmsg, rosservice, rossrv, rosparam, rosed, catkin, cd, 
		rosbag, roslaunch, source},
	morecomment=[l]{\#},
	morestring=[b]',
	morestring=[b]",
	sensitive=true,
	morecomment=[l]{\#},
}

\lstdefinelanguage{plainlang}{
	keywords={},
	morecomment=[l]{//},
	morecomment=[s]{/*}{*/},
	morestring=[b]',
	morestring=[b]",
}

\lstdefinelanguage
[x64]{Assembler}     % add a "x64" dialect of Assembler
[x86masm]{Assembler} % based on the "x86masm" dialect
% with these extra keywords:
{morekeywords={CDQE,CQO,CMPSQ,CMPXCHG16B,JRCXZ,LODSQ,MOVSXD, %
		POPFQ,PUSHFQ,SCASQ,STOSQ,IRETQ,RDTSCP,SWAPGS, %
		rax,rdx,rcx,rbx,rsi,rdi,rsp,rbp, %
		r8,r8d,r8w,r8b,r9,r9d,r9w,r9b, %
		r10,r10d,r10w,r10b,r11,r11d,r11w,r11b, %
		r12,r12d,r12w,r12b,r13,r13d,r13w,r13b, %
		r14,r14d,r14w,r14b,r15,r15d,r15w,r15b}} % etc.

\lstdefinelanguage{Cuda}{
	language=C++,
	morekeywords={
		__global__, __shared__, __device__, __host__,
		__syncthreads,
		dim3
	},
	moredelim=[s][\ttfamily \color{green!60!black}]{<<<}{>>>},
}

\lstdefinelanguage{JavaScript}{
	keywords={break, case, catch, continue, debugger, default, delete, do, else, finally, for, function, if, in, instanceof, new, return, switch, this, throw, try, typeof, var, void, while, with},
	morecomment=[l]{//},
	morecomment=[s]{/*}{*/},
	morestring=[b]',
	morestring=[b]",
	sensitive=true
}

\lstdefinelanguage{Scheme}{
	keywords={define, lambda},
	morekeywords=[1]{define, define-syntax, define-macro, lambda, define-stream, stream-lambda},
	morekeywords=[2]{begin, call-with-current-continuation, call/cc,
		call-with-input-file, call-with-output-file, case, cond,
		do, else, for-each, if,
		let*, let, let-syntax, letrec, letrec-syntax,
		let-values, let*-values,
		and, or, not, delay, force,
		quasiquote, quote, unquote, unquote-splicing,
		map, fold, syntax, syntax-rules, eval, environment, query, car, cdr, cons, set!, set-car!, set-cdr!},
	morekeywords=[3]{import, export},
	morecomment=[l]{;},
	morecomment=[s]{\#|}{|\#},
	alsodigit=!\$\%&*+-./:<=>?@^_~,
	literate=*{`}{{`}}{1},
	sensitive=true
}

\lstdefinestyle{basic}{
	breakatwhitespace=false,         % sets if automatic breaks should only happen at whitespace
	breaklines=true,                 % sets automatic line breaking
	captionpos=b,                    % sets the caption-position to bottom
	escapeinside={\%*}{*)},          % if you want to add LaTeX within your code
	extendedchars=true,              % lets you use non-ASCII characters; for 8-bits encodings only, does not work with UTF-8
	frame=all,	                 	 % adds a frame around the code
	keepspaces=true,                 % keeps spaces in text, useful for keeping indentation of code (possibly needs columns=flexible)
	language=bash,                 	 % the language of the code
	numbers=left,                    % where to put the line-numbers; possible values are (none, left, right)
	numbersep=5pt,                   % how far the line-numbers are from the code
	showspaces=false,                % show spaces everywhere adding particular underscores; it overrides 'showstringspaces'
	showstringspaces=false,          % underline spaces within strings only
	showtabs=false,                  % show tabs within strings adding particular underscores
	stepnumber=1,                    % the step between two line-numbers. If it's 1, each line will be numbered
	tabsize=2,	                     % sets default tabsize to 2 spaces
	title=\lstname                   % show the filename of files included with \lstinputlisting; also try caption instead of title
}
\lstdefinestyle{dark}{
	style=basic,
	backgroundcolor=\color{black!80!white},   % choose the background color; you must add \usepackage{color} or \usepackage{xcolor}; should come as last argument
	basicstyle=\ttfamily \color{white},        % the size of the fonts that are used for the code
	commentstyle=\color{gray},      % comment style
	keywordstyle=\color{green!50!white},     % keyword style
	numberstyle=\tiny\color{gray},   % the style that is used for the line-numbers
	rulecolor=\color{black!50},         % if not set, the frame-color may be changed on line-breaks within not-black text (e.g. comments (green here))
	stringstyle=\color{blue!50!white},        % string literal style
}
\lstdefinestyle{light}{
	style=basic,
	backgroundcolor=\color{white},   		% choose the background color; you must add \usepackage{color} or \usepackage{xcolor}; should come as last argument
	basicstyle=\footnotesize,        		% the size of the fonts that are used for the code
	commentstyle=\color{green!60!black},    % comment style
	keywordstyle=\color{violet},     		% keyword style
	numberstyle=\tiny\color{gray},   		% the style that is used for the line-numbers
	rulecolor=\color{black!50},      		% if not set, the frame-color may be changed on line-breaks within not-black text (e.g. comments (green here))
	stringstyle=\color{blue},        		% string literal style
}

\lstdefinestyle{plain}{
	style=light,
	language=plainlang
}

\lstdefinestyle{bash}{
	style=dark,
	language=bash,
	morekeywords={sudo, get, apt, source, install, mkdir, cd, snap},             % if you want to add more keywords to the set
}

\lstdefinestyle{BNF}{
	style=light,
	language=BNF
}

\lstdefinestyle{C++}{
	style=light,
	language=C++
}

\lstdefinestyle{Cuda}{
	style=light,
	language=Cuda
}

\lstdefinestyle{CSharp}{
	style=light,
	language=[Sharp]C
}

\lstdefinestyle{html}{
	style=light,
	language=html
}

\lstdefinestyle{Java}{
	style=light,
	language=Java
}

\lstdefinestyle{LaTeX}{
	style=light,
	language=[LaTeX]TeX
}

\lstdefinestyle{make}{
	style=light,
	language=make
}

\lstdefinestyle{Pascal}{
	style=light,
	language=Pascal
}

\lstdefinestyle{Perl}{
	style=light,
	language=Perl
}

\lstdefinestyle{PHP}{
	style=light,
	language=PHP
}

\lstdefinestyle{PostScript}{
	style=light,
	language=PostScript
}

\lstdefinestyle{Prolog}{
	style=light,
	language=Prolog
}

\lstdefinestyle{Python}{
	style=light,
	language=Python
}

\lstdefinestyle{R}{
	style=light,
	language=R
}

\lstdefinestyle{roscmd}{
	style=dark,
	language=ROSBASH
}

\lstdefinestyle{Ruby}{
	style=light,
	language=Ruby
}

\lstdefinestyle{SQL}{
	style=light,
	language=SQL
}

\lstdefinestyle{tcl}{
	style=light,
	language=tcl
}

\lstdefinestyle{Verilog}{
	style=light,
	language=Verilog
}

\lstdefinestyle{XML}{
	style=light,
	language=XML
}